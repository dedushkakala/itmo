\documentclass[12pt, a4paper]{scrartcl}
\usepackage [utf8] {inputenc} 
\usepackage [english,russian] {babel}
\usepackage{indentfirst}
\usepackage{misccorr}
\usepackage{graphicx}
\usepackage{amsmath}
\usepackage [warn] { mathtext }
\usepackage{minted} % for highlight


\begin{document}
	\section{Процессы и потоки}
		\textbf{Процесс} - экземпляр замущенного исполняемого файла. Нужен для того, что абстрагивровать писание программы, как будто все возможности только твои.
		
		
		Процесс состоит из \textbf{потоков}. Это вычислительный контекст. Раньше он обычно был один, потому что у нас было только одно ядро. Но со временем появились ядра и появилась необходимость писать программу в несколько потоков.
		
		\subsection{Что общего у потоков?}
			\begin{enumerate}
				\item регистры	
				\item стек
				\item ид потока
				\item IPC
			\end{enumerate}
		\subsection{Как выглядит память у потока?}
		stack - сохранение регистров\par
		| \par
		V \par
		..... \par
	 	|\par
		heap - в с++(free store) - динамическая память\par
		\_\_\_\_\_\par
		data - статик переменные\par
		\_\_\_\_\_\par
		text - код программы\par
		\subsection{Действия с процессами}
			\subsubsection{fork}
				\mintinline{c++}{pid_t fork(void);} 
				
				Это создание нового процесса. До форка был один процесс(1). После стало два процесса и все стало параллельно 
				
				Пример кода для использования:
				\begin{minted}[linenos, frame=lines, framesep=2mm, tabsize = 4, breaklines]{c++}
				const pit_t pid = fork();
				if (pid == -1) {
					// handle error
				}
				if (!pid) {
					// we are child - do child work
				}
				if (pid) {
					// we are parent - do parent work
				}
				\end{minted}
				
				Но понятное дело, что писать так \mintinline{c++}{if(pid)} постоянно грустно и произойдет \textit{fork bomba}
			\subsubsection{execve}
				\mintinline{c++}{int execve(const char *filename, char *const argv[], char *const envp[]);}
				Заместить наш процесс новым процесс по имени \mintinline{c++}{filename} c аргументами \mintinline{c++}{args}
				
				Пример - мы хотим вызвать man из bash-a
				
				\begin{minted}[linenos, frame=lines, framesep=2mm, tabsize = 4, breaklines]{c++}
				const pit_t pid = fork(); // сделали форк основного процесса
				if (pid) {
					// ... parent - continue working with bash
				}
				else (!pid) {
					execve("man", "execve"); // вызвали процесс man 
				}
				\end{minted}
			\subsubsection{waitpid}
				\mintinline{c++}{pid_t waitpid(pid_t pid, int *status, int options);}
				
				Подождать конец процесса с ид \mintinline{c++}{pid}.
				Например в предыдущем коде можно внутри \mintinline{c++}{if(pid)} написать waitpid - подождать пока выйдет ребенок
			\subsubsection{Exit and kill}
				\mintinline{c++}{void exit(int status);}\par
				\mintinline{c++}{int kill(pid_t pid, int sig);}\par
				Убить себя или чужой процесс с состоянием \mintinline{c++}{status(pid)]}. Если хочешь завершить чужой процесс - передаешь \mintinline{c++}{sig} с которым ты хочешь завершить процесс
		\subsection	{Виды процессов}
			\subsubsection {Процесс 1}
				Ядро всегда создает процесс с номером 1. У нее есть две специальные команды для системы инициализации. Примером может быть \textbf{init/systemd}. Они запускают системные сервисы.
			\subsubsection {Zombie процессы}
				Что будет если не сделать \textbf{waitpid} у родителя? После завершения процесса он освобождается, но если не вызвать он будет "зомби-процессом" - будет занимать немного места, но он будет щанимать ид-шники. Также его нельзя убить.
			\subsection{Orphan(сирота)}
				Ситуация когда родителя убили. Пид родителя становится 1.
		\subsection{Scheduler}
			Как процессы живут вместе? Давайте постепенно переключать процессы. Этим занимается ОС с помощью scheduler. Он берет и с помощью некоторого порядка задает очередь выполнения процессов. Каждый процесс работает по определенному косучку времени(например 15 мс).
			
				
				
\end{document}