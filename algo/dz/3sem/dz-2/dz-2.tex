\documentclass[12pt, a4paper]{scrartcl}
\usepackage [utf8] {inputenc} 
\usepackage [english,russian] {babel}
\usepackage{indentfirst}
\usepackage{misccorr}
\usepackage{graphicx}
\usepackage{amsmath}
\usepackage [warn] { mathtext }

\begin{document}
	\LARGE{\textbf{Федер Евгений, Домашнее задание №1}}\par
	
	%%%%%%%%%%%%%%%%%%%%%%%%%%
	\emph{\textbf{Задание 1.}}\par
	
	Возьмем рандомную вершину и зафиксируем ее цвет. Нам известно, что на каждом ребре должны быть вершины разного цвета, поэтому просто берем и красим всех соседей текущей вершины в противоположный цвет и делаем тоже самое для них.\par
	Почему может быть не 2 раскраски. Такой случай достигается когда у нас сосед вершины уже покрашен и покрашен ровно в тот цвет, который у текущей вершины. Тогда получится 0 раскрасок.(кольцо из 3 вершин, ну или есть именно цикл нечетной длины)\par
	А почему 2? Каждая вершина может быть двух цветов и одна вершина точно задает всю раскраску. Поэтому мы фиксируем для начальной вершины два цвета и получаем два способа.\par
	Одной раскраски не может быть(пусть она есть, то инвертируем все цвета - получим новую раскраску).
	%%%%%%%%%%%%%%%%%%%%%%%%%%%
	\newpage\emph{\textbf{Задание 2.}}\par
	
	Сделаем чтобы какой-то из двух цветов был $true$, а второй $false$(можно ребро только когда значения разные). Получаем, что нам для каждой вершины нужно сделать две вершины со значением $true$ и $false$ и будем строить соответству.щие ребра между $true1-false2$ и $true2-false1$. Собственно это и есть задача 2-sat.
\end{document}