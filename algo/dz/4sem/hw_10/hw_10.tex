\documentclass[12pt, a4paper]{scrartcl}
\usepackage [utf8] {inputenc} 
\usepackage [english,russian] {babel}
\usepackage{indentfirst}
\usepackage{misccorr}
\usepackage{graphicx}
\usepackage{amsmath}
\usepackage [warn] { mathtext }

\begin{document}
	\LARGE{\textbf{Федер Евгений, Домашнее задание №10}}\par
	
	%%%%%%%%%%%%%%%%%%%%%%%%%%
	\emph{\textbf{Задание 1.}}\par
	Воспользуемся 2 заданием \\
	$gcd(F_n, F_{n - 1}) = gcd(F_n - F_{n - 1}, F_{n - 1}) = gcd(F_{n - 2}, F_{n - 1}) = gcd(F_{n - 1}, F_{n - 2})$ \\
	Таким образом мы спускаемся до $F_2$ и $F_1$, а их НОД равен 1.
	
	\emph{\textbf{Задание 2.}}\par
		Докажем утверждение сразу в две стороны.\\
		Когда $a == b$ случай очевиден. Все немного ломается. 0 - не натуральное число\\
		Теперь рассмотрим $a < b$.\\
		Покажем $gcd(a, b) = gcd(a, b \pm a)$ 
		\begin{enumerate}
			\item Так как a делится на gcd, и b делится на gcd, то $(a \pm b)$ делится на gcd.(вынести общий множитель)
			\item Из 1 пункта, $gcd(a, a \pm b) \leq gcd(a, b)$
			\item Больше не может быть, так как тогда $\exists x : x * gcd(a, b)$ делит и $a$ и на $a \pm b$ а значит делит и $b$
			\item Следовательно они равны.
		\end{enumerate}
		Отлично, из этого следует утверждение в обе стороны:
		\begin{itemize}
			\item Влево: возьмем $a$ и $b$, вычтем из $b$ $a$, и получим равенство.
			\item Также можно сделать и для вправо(прибавить $a$)
		\end{itemize}
	\emph{\textbf{Задание 3.}}\par
		$d_x$ кратно $x$
		\begin{itemize}
			\item $\delta(a) * \delta(b) = \sum d_a^k * \sum d_b^k = \sum (d_a * d_b) ^ k$
			\item Пусть у нас есть $d_ab$, тогда разделим ее на множители и каждый множитель принадлежит или $a$ или $b$. Следовательно $\delta(a * b) = \sum d_{a * b}^k = \sum (d_a * d_b) ^ k$ 
			\item приравниваем оба пункта и ЧТД
		\end{itemize}
	\emph{\textbf{Задание 4.}}\par
		\begin{itemize}
			\item $p*q = x \Rightarrow p = x / q$
			\item $\phi(p * q) = (p - 1) * (q - 1) = (x / q - 1) * (q - 1) = x - x / q - q + 1$
			\item $x - x / q - q + 1 = y \Rightarrow x * q - x - q ^ 2 + q = q * y \Rightarrow q^2 + q * (y - x - 1) + x = 0$
			\item Решаем уравнение и проверяем. Если получилось - мы победили. Если нет - таких не существует.
		\end{itemize}
\end{document}